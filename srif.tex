\documentclass{acm_proc_article-sp}
\usepackage{hyperref}

\newfont{\mycrnotice}{ptmr8t at 7pt}
\newfont{\myconfname}{ptmri8t at 7pt}
\let\crnotice\mycrnotice%
\let\confname\myconfname%

\permission{Permission to make digital or hard copies of all or part of this work for personal or classroom use is granted without fee provided that copies are not made or distributed for profit or commercial advantage and that copies bear this notice and the full citation on the first page. Copyrights for components of this work owned by others than the author(s) must be honored. Abstracting with credit is permitted. To copy otherwise, or republish, to post on servers or to redistribute to lists, requires prior specific permission and/or a fee. Request permissions from Permissions@acm.org.}
\conferenceinfo{SRIF'15,}{September 7, 2015, Paris, France. \\
{\mycrnotice{Copyright is held by the owner/author(s). Publication rights licensed to ACM.}}}
\copyrightetc{ACM \the\acmcopyr}
\crdata{ISBN 978-1-4503-3532-4/15/09\ ...\$15.00.\\
DOI: http://dx.doi.org/10.1145/2801676.2801684}

\clubpenalty=10000
\widowpenalty = 10000
\interfootnotelinepenalty=1000

\begin{document}
\title{Demo: IIO Blocks for GNU Radio}
\subtitle{[Extended Abstract]}

\numberofauthors{1}

\author{
\alignauthor
Paul Cercueil\\
	\affaddr{Analog Devices Inc.}\\
	\affaddr{Wilhelm-Wagenfeld-Strasse 6}\\
	\affaddr{80807 Munich, Germany}\\
	\email{paul.cercueil@analog.com}
}

\maketitle

\keywords{Linux, IIO, SDR, GNU Radio}

\section{The IIO framework of the Linux kernel}

The Industrial Input / Output (IIO) framework has been in
the upstream Linux kernels since 2011, and is responsible for handling sensors,
converters, integrated transceivers and other real-world I/O devices. It provides
a hardware abstraction layer with a consistent API for the user space
applications.

As of today, the IIO framework consists of approximatively 200 drivers,
of which many handle multiple devices.
It supports discrete components as well as
integrated transceivers like the Analog Devices AD9361, a 2 x 2 RF Agile Transceiver, found in many SDR products.

\section{The IIO blocks for GNU Radio}

GNU Radio\footnote{\url{http://gnuradio.org}}
is a free and open-source graphical programming environment, that can
be used to design software-defined radios (SDR) using a large panel of signal
processing blocks. It is widely used in hobbyist, academic, commercial and
military environments to support real-world radio systems as well as research
and development in wireless communications.

Recently, Analog Devices Inc. (ADI) created several new blocks for GNU Radio
that can be used to stream samples from and to high-speed IIO
devices\footnote{\url{https://github.com/analogdevicesinc/gnuradio}}.
Adding a few signal processing
blocks, it becomes very easy to create complex applications such as
LTE or IEEE 802.11g OFDM receivers, radar processing, or just simple
applications like wide-band FM transceivers.

\begin{figure}[htbp]
\centering
\includegraphics[scale=0.25]{libiio.pdf}
\caption{From the HW to GNU Radio with libiio}
\end{figure}

These new GNU Radio blocks are built around ADI's
libiio\footnote{\url{http://wiki.analog.com/resources/tools-software/linux-software/libiio}}
library, which concentrates most of the complexity,
supports all IIO devices and has built-in support
for data streaming over the network.

This one particular feature
has several benefits; it allows the GNU Radio flowgraphs to run
on other operating systems than Linux (including Windows),
and on more capable computers than the development board
where the IIO devices are attached, which is usually not
suitable for high-performance and real-time data processing.

\section{Demo: 802.11g WiFi packet sniffer}

This demo will focus on presenting the IIO blocks for GNU Radio; how they can be
used, their functionalities, the controls they provide over the hardware.
For this purpose, the demo will be a GNU Radio flowgraph that corresponds to a
IEEE 802.11g (WiFi) packet sniffer, which has the advantage of showing a
real-world use case of a fast (20 MHz sample rate) software defined radio
application, while remaining relatively simple with only a handful of blocks.

This demo will use the
\texttt{gr-ieee802-11}\footnote{\url{https://github.com/bastibl/gr-ieee802-11}}
blocks for GNU Radio. The hardware will be composed of an off-the-shelf low-cost
FPGA development board (ZedBoard) together with a AD-FMCOMMS3 (AD9361) FMC card.

\balancecolumns
\end{document}
