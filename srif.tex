\documentclass{acm_proc_article-sp}
\usepackage{hyperref}
%\usepackage{graphicx}

\interfootnotelinepenalty=1000

\begin{document}
\title{Demo: IIO Blocks for GNU Radio}
\subtitle{[Extended Abstract]}

\numberofauthors{1}

\author{
\alignauthor
Paul Cercueil\\
	\affaddr{Analog Devices Inc.}\\
	\affaddr{Wilhelm-Wagenfeld-Strasse 6}\\
	\affaddr{80807 Munich, Germany}\\
	\email{paul.cercueil@analog.com}
}

\maketitle

\keywords{Linux, IIO, SDR, GNU Radio}

%\begin{abstract}
%This is the abstract
%\end{abstract}

\section{The IIO subsystem of the Linux kernel}

The Industrial Input / Output (IIO) framework of the Linux kernel has been in
the upstream Linux kernels since 2011, and is responsible for handling sensors,
converters, integrated transceivers and other real word I/O devices. It provides
a hardware abstraction layer with a consistent API for the user space
applications.

As of today, the IIO framework consists of ~200 drivers, many supporting
different devices. It supports discrete ADCs, DACs, VGAs, PLLs... which
can be used to build a discrete SDR solution. It also supports integrated
transceivers like the AD9361, found in many SDR products.

\section{The IIO blocks for GNU Radio}

GNU Radio\footnote{\url{http://gnuradio.org}}
is a free and open-source graphical programming environment, that can
be used to design software-defined radios (SDR) using a large panel of signal
processing blocks. It is widely used in hobbyist, academic, commercial and
military environments to support real-world radio systems as well as research
and development in wireless communications.

Recently, Analog Devices created several new blocks for GNU Radio that can be
used to stream samples from and to high-speed Industrial I/O (IIO)
devices\footnote{\url{https://github.com/analogdevicesinc/gnuradio/tree/gr-iio}}.
Adding a few signal processing
blocks, it becomes very easy to create complex applications such as
LTE or IEEE 802.11g OFDM receivers, radar processing, or just simple
applications like wide-band FM transceivers.

\begin{figure}[htbp]
\centering
\includegraphics[scale=0.25]{libiio.pdf}
\caption{From the HW to GNU Radio with libiio}
\end{figure}

Those new GNU Radio blocks use the
libiio\footnote{\url{http://wiki.analog.com/resources/tools-software/linux-software/libiio}}
library as their base; as a result,
they inherit some features from the library, like the ability to connect
remotely to the target platform over the network.

This has several benefits, including the possibility to run them on other
operating systems than Linux (including Windows), but it also permits to
offload the heavy computations performed by GNU Radio to a more capable computer,
in order to reach higher sample rates.

\section{Flowgraph example: 802.11g WiFi packet sniffer}

This demo will focus on presenting the IIO blocks for GNU Radio; how they can be
used, their functionalities, the controls they provide over the hardware.
For this purpose, the demo will be a GNU Radio flowgraph that corresponds to a
IEEE 802.11g (WiFi) packet sniffer, which has the advantage of showing a
real-world use case of a fast (20 MHz sample rate) software defined radio
application, while remaining relatively simple with only a handful of blocks.

This demo will use the
\texttt{gr-ieee802-11}\footnote{\url{https://github.com/bastibl/gr-ieee802-11}}
blocks for GNU Radio. The hardware will be composed of an off-the-shelf low-cost
FPGA development board (Zynq Zed) together with a AD-FMCOMMS3 (AD9361) FMC card.

\balancecolumns
\end{document}
